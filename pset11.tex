\documentclass[a4paper]{exam}

\usepackage{amsfonts,amsmath,amsthm}
\usepackage{geometry}

\title{Problem Set 11: Mathematical Induction}
\author{CS/MATH 113 Discrete Mathematics}
\date{Spring 2024}

\boxedpoints

\printanswers

\begin{document}
\maketitle

\begin{questions}
\question Prove the following using mathematical induction on $n\in\mathbb{Z}$ where $n>0$.
  \begin{parts}
  \part  $\sum_{r=1}^{n} r(r + 1) = \frac{1}{3} \cdot n(n + 1)(n + 2)$
    \begin{solution}
      We have to prove $P(n): \sum_{r=1}^{n} r(r + 1) = \frac{1}{3} \cdot n(n + 1)(n + 2)$.\\
      We provide a proof by mathematical induction.
      \begin{proof}$P(n)$ is true for all integers greater than $0$.
        
      \underline{Base case}: $P(1)$:
        \begin{align*}
          1(1+1) & = \frac{1}{3}  1(1 + 1)(1 + 2)\\
          \implies 1(2) & = \frac{1}{3}  1(2)(3)\\
          \implies 2 & = 2
      \end{align*}
      $P(1)$ is true. The base case for the proof by mathematical induction holds.

      \underline{Inductive step}:\\
        \begin{align*}
          & IH: P(k): \sum_{r=1}^{k} r(r + 1) = \frac{1}{3}  k(k + 1)(k + 2)\\
          & \implies \sum_{r=1}^{k} r(r + 1) + (k+1)(k+2) = \frac{1}{3}  k(k + 1)(k + 2) + (k+1)(k+2)\\
          & \implies \sum_{r=1}^{k+1} r(r + 1) = (k + 1)(k + 2)(\frac{k}{3}  + 1) = \frac{1}{3}(k + 1)(k + 2)(k+3)\\
          & \implies P(k+1)
        \end{align*}
        This completes the inductive step and the proof of $P(n)$ by mathematical induction.
      \end{proof}
    \end{solution}
    
  \part $1^5 + 2^5 + 3^5 + \ldots + n^5 = \frac{1}{12}  n^2 (n + 1)^2 (2n^2 + 2n - 1)$
    \begin{solution}
      We have to prove $P(n): \sum_{i=1}^n i^5 = \frac{1}{12}  n^2 (n + 1)^2 (2n^2 + 2n - 1)$.\\
      We provide a proof by mathematical induction.
      \begin{proof}$P(n)$ is true for all integers greater than $0$.
        
      \underline{Base case}: $P(1)$:
        \begin{align*}
          1^5 & = \frac{1}{12}  1^2 (1 + 1)^2 (2(1)^2 + 2(1) - 1)\\
          \implies 1^5 & = \frac{1}{12}  1 (2)^2 (2 + 2 - 1)\\
          \implies 1 & = \frac{1}{12}  1 (4) (3)\\
          \implies 1 & = 1
      \end{align*}
      $P(1)$ is true. The base case for the proof by mathematical induction holds.

      \underline{Inductive step}:\\
      \begin{align*}
        & IH: P(k): \sum_{i=1}^k i^5 = \frac{1}{12}  k^2 (k + 1)^2 (2k^2 + 2k - 1)\\
        & \implies \sum_{i=1}^k i^5 + (k+1)^5= \frac{1}{12}  k^2 (k + 1)^2 (2k^2 + 2k - 1) + (k+1)^5\\
        & \implies \sum_{i=1}^{k+1} i^5= \frac{1}{12}(k+1)^2(k^2 (2k^2 + 2k - 1) + 12(k+1)^3)\\
        & \implies \sum_{i=1}^{k+1} i^5= \ldots = \frac{1}{12}(k+1)^2(2k^4 + 14k^3 + 35k^2 + 36k + 12)
        \end{align*}
        It can be shown that $P(k+1)$ reduces to the expression above.\\
        $\implies P(k)\implies P(k+1)$.\\
        This completes the inductive step and the proof of $P(n)$ by mathematical induction.
      \end{proof}
    \end{solution}

  \part $5^{2n+1} + 2^{2n+1}$ is divisible by 7.
    \begin{solution}
      We have to prove $P(n): 5^{2n+1} + 2^{2n+1} = 7m$ for some integer $m$.\\
      We provide a proof by mathematical induction.
      \begin{proof}$P(n)$ is true for all integers greater than $0$.
        
      \underline{Base case}: $P(1)$:\\
        \begin{align*}
          5^{2(1)+1} + 2^{2(1)+1} &= 7m\\
          \implies 5^{3} + 2^{3} &= 7m\\
          \implies 125 + 8 &= 7m\\
          \implies 133 &= 7m
      \end{align*}
      $P(1)$ is true when $m=19$.\\
      The base case for the proof by mathematical induction holds.

      \underline{Inductive step}:\\
        LHS of $P(k+1)$:
        \begin{align*}
          5^{2(k+1)+1} + 2^{2(k+1)+1} &= 5^2(5^{2k+1}) + 2^{2k+3}\\
                                      &=^{IH}  5^2(7m-2^{2k+1}) + 2^{2k+3}\\
                                      &=  5^27m - 5^22^{2k+1} + 2^{2k+3}\\
                                      &=  5^27m - 5^22^{2k+1} + 2^{2k+3}\\
                                      &=  5^27m - 2^{2k+1}(5^2 - 2^2)\\
                                      &=  5^27m - 2^{2k+1}(25 - 4)\\
                                      &=  5^27m - 2^{2k+1}(21)\\
                                      &=  5^27m - 2^{2k+1}(7\cdot3)\\
                                      &=  7(5^2m - 2^{2k+1}3)\\
                                      &=  7m_1
        \end{align*}
        This completes the inductive step and the proof of $P(n)$ by mathematical induction.
      \end{proof}
    \end{solution}

    
  \part $a^{2^n} - 1$ is divisible by $8 \times 2^{n-1}$ for all odd integers $a$.
    \begin{solution}
      We have to prove $P(n): a^{2^n} - 1 = 2^{n+2}m$ for some integer $m$.\\
      We provide a proof by mathematical induction.
      \begin{proof} $P(n)$ is true for all integers greater than $0$.
        
        We denote the odd integer as $a=2i+1$ for some integer, $i$.
      \underline{Base case}: $P(1)$:\\
      \begin{align*}
        (2i+1)^{2^1} -1 &= 2^{1+2}m\\
        \implies 4i^2+4i+1-1 &= 2^3m\\
        \implies 4i^2+4i &= 8m\\
        \implies 4(i^2+i) &= 8m
        \end{align*}
        Now, $i$ may be odd or even.\\
        If $i$ is odd, then $i^2$ is odd and $i^2+i$ is even.\\
        Otherwise, if $i$ is even, then $i^2$ is even and $i^2+i$ is even.\\
        In each case, $i^2+i$ is even. That is, $i^2+i=2p$ for some integer, $p$.\\
        Continuing for $P(1)$ from above,
      \begin{align*}
        4(i^2+i) &= 8m\\
        \implies 4(2p) &= 8m\\
        \implies 8p &= 8m
        \end{align*}
        $P(1)$ is true when $m=p$.\\
        The base case for the proof by mathematical induction holds.

      \underline{Inductive step}:\\
        $P(k+1)$:
        \begin{align*}
          (2i+1)^{2^{k+1}}-1 &= ((2i+1)^{2^k})^2-1\\
                             &=^{IH} (2^{k+2}m + 1)^2-1\\
                             &= (2^{2(k+2)}m^2 +2^{k+3}m + 1)-1\\
                             &= 2^{2k+4}m^2 +2^{k+3}m\\
                             &= 2^{k+3}(2^{k+1}m^2 +m)\\
                             &= 2^{k+3}m_1
        \end{align*}
        This completes the inductive step and the proof of $P(n)$ by mathematical induction.
      \end{proof}
    \end{solution}
    
  \part if $A_1, A_2,\ldots,A_n$ and $B$ are sets, then
    \[
      (A_1-B)\cup(A_2-B)\cup\ldots\cup(A_n-B)=(A_1\cup A_2\cup\ldots \cup A_n)-B
    \]
    \begin{solution}
      We have to prove $P(n): (A_1-B)\cup(A_2-B)\cup\ldots\cup(A_n-B)=(A_1\cup A_2\cup\ldots \cup A_n)-B$ for some integer $m$.\\
      We provide a proof by mathematical induction.
      \begin{proof} $P(n)$ is true for all integers greater than $0$.

      \underline{Base case}: $P(1)$:\\
      \begin{align*}
        (A_1-B) &= (A_1)-B\\
        \implies A_1-B &= A_1-B
        \end{align*}
        $P(1)$ is true.\\
        The base case for the proof by mathematical induction holds.

      \underline{Inductive step}:\\
        Starting with LHS of $P(k+1)$:
          \begin{multline*}
          (A_1-B)\cup(A_2-B)\cup\ldots\cup(A_k-B)\cup(A_{k+1}-B)\\
          =^{IH} ((A_1\cup A_2\cup\ldots \cup A_n)-B) \cup(A_{k+1}-B)\hfill\\
          = \{ x\mid (x\in A_1 \lor x\in A_2 \lor\ldots\lor x\in A_k) \land x\not\in B\} \cup\{ x\mid (x\in A_{k+1} \land x\not\in B\}\hfill \\
          = \{ x\mid ((x\in A_1 \lor x\in A_2 \lor\ldots\lor x\in A_k) \land x\not\in B)\lor (x\in A_{k+1} \land x\not\in B)\}\hfill \\
          = \{ x\mid (x\in A_1 \lor x\in A_2 \lor\ldots\lor x\in A_k\lor x\in A_{k+1}) \land x\not\in B\}\hfill \\
          = (A_1 \cup x\in A_2 \cup\ldots\cup A_k\cup A_{k+1}) - B \hfill
        \end{multline*}
        This completes the inductive step and the proof of $P(n)$ by mathematical induction.
      \end{proof}
    \end{solution}
  \end{parts}

\question For which non-negative integers $n$ does the property $n^2 \le n!$ hold? Prove your answer using mathematical induction.
  \begin{solution}
    We are given the statement $P(n): n^2\le n!$.\\
    The non-negative values that it holds for are $n = 0, 1, 4, 5, 6, 7, 8, \ldots$.\\
    We provide direct proofs for $n=0$ and $n=1$ and a proof by mathematical induction for $n\ge 4$.

    \begin{proof} $P(n): n^2\le n!$ is true for $n=0,1,4,5,6,7,\ldots$.
      \begin{align*}
        P(0): & & P(1):\\
        0^2  \le 0! && 1^2 \le 1!\\
        \implies 0  \le 1 && \implies 1 \le 1
      \end{align*}
      $P(0)$ and $P(1)$ are true.

      \underline{Base case}: $P(4)$:\\
      \begin{align*}
        4^2  & \le 4!\\
        \implies 16  &\le 24
      \end{align*}
        $P(4)$ is true.\\
        The base case for the proof by mathematical induction holds.

      \underline{Inductive step}:\\
      Consider IH: $P(k): k^2\le k!$ or
      \begin{align*}
        k!\ge k^2 &\implies (k+1)k! \ge (k+1)k^2\\
                  &\implies (k+1)! \ge (k+1)k^2\\
                  &\implies (k+1)! \ge (k+1)k^2 \ge (k+1)(k+1)\\
                  &\implies (k+1)! \ge (k+1)^2
      \end{align*}
      This completes the inductive step and the proof of $P(n)$ by mathematical induction.
      \end{proof}
  \end{solution}

\question We are familiar with the Fibonacci numbers, defined as $f_{n+1} = f_n + f_{n-1}, f_0=1, f_1=1$ for all $n \geq 1$. The Fibonacci numbers turn out to have several interesting properties. Prove each of the following by mathematical induction.
  \begin{parts}
  \part $f_1 + f_2 + f_3 + \ldots + f_n = f_{n+2} - 1$.
    \begin{solution}
      We are given the statement $P(n): \sum_{i=1}^n f_i = f_{n+2} - 1$.\\
      Implicitly, we see that $n\ge 1$.\\
      We provide a proof by mathematical induction for $P(n)$ over the domain.

    \begin{proof} $P(n)$ is true for $n=\ge 1$.

      \underline{Base case}: $P(1)$:
      \begin{align*}
        f_1 &= f_{1+2} - 1\\
        \implies 1 &= f_{3} - 1\\
        \implies 1 &= 2 - 1\\
        \implies 1 &= 1
      \end{align*}
        $P(1)$ is true.\\
        The base case for the proof by mathematical induction holds.

      \underline{Inductive step}:\\
      \begin{align*}
        IH: P(k): \sum_{i=1}^k f_i = f_{k+2} - 1 &\implies \sum_{i=1}^k f_i  +f_{k+1}= f_{k+2} - 1 +f_{k+1}\\
                                           &\implies \sum_{i=1}^{k+1} f_i= f_{k+2} +f_{k+1} - 1\\
                                           &\implies \sum_{i=1}^{k+1} f_i= f_{k+3} - 1\\
                                           &\implies P(k+1)
      \end{align*}
      This completes the inductive step and the proof of $P(n)$ by mathematical induction.
      \end{proof}
    \end{solution}

  \part $f_2 + f_4 + f_6 + \ldots + f_{2n} = f_{2n+1} - 1$.
    \begin{solution}
      We are given the statement $P(n): \sum_{i=1}^n f_{2i} = f_{2n+1} - 1$.\\
      Implicitly, we see that $n\ge 1$.\\
      We provide a proof by mathematical induction for $P(n)$ over the domain.

    \begin{proof} $P(n)$ is true for $n=\ge 1$.

      \underline{Base case}: $P(1)$:
      \begin{align*}
        f_{2(1)} &= f_{2(1)+1} - 1\\
        \implies f_2 &= f_3 - 1\\
        \implies 1 &= 2 - 1\\
        \implies 1 &= 1
      \end{align*}
        $P(1)$ is true.\\
        The base case for the proof by mathematical induction holds.

      \underline{Inductive step}:\\
      \begin{align*}
        IH: P(k): \sum_{i=1}^k f_{2i} = f_{2k+1} - 1 &\implies \sum_{i=1}^k f_{2i}  +f_{2(k+1)}= f_{2k+1} - 1 +f_{2(k+1)}\\
                                                     &\implies \sum_{i=1}^{k+1} f_{2i} = f_{2k+1} +f_{2k+2} - 1\\
                                                     &\implies \sum_{i=1}^{k+1} f_{2i}= f_{2k+3} - 1 = f_{2(k+1)+1} - 1\\
                                                     &\implies P(k+1)
      \end{align*}
      This completes the inductive step and the proof of $P(n)$ by mathematical induction.
      \end{proof}
    \end{solution}
  \end{parts}

\question A \textit{convex polygon} is a polygon in which all the interior angles are smaller than $\pi$ radians. An $n$-gon is a polygon with $n$ sides. Give a proof by mathematical induction that the sum of the interior angles of a convex $n$-gon is $(n-2)\pi$ where $n\ge 3$. 

  \textit{Hint}: A convex $n$-gon can be divided into a convex $(n-1)$-gon and a triangle.
  \begin{solution}
    % Enter your solution here.
  \end{solution}
  

\end{questions}
\end{document}
%%% Local Variables:
%%% mode: latex
%%% TeX-master: t
%%% End: