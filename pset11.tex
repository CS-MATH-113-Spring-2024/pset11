\documentclass[a4paper]{exam}

\usepackage{geometry}
\usepackage{amsmath}
\usepackage{amsfonts}

\title{Problem Set 11: Mathematical Induction}
\author{CS/MATH 113 Discrete Mathematics}
\date{Spring 2024}

\boxedpoints

\printanswers

\begin{document}
\maketitle

\begin{questions}
\question Prove the following using mathematical induction on $n\in\mathbb{Z}$ where $n>0$.
  \begin{parts}
  \part  $\sum_{r=1}^{n} r(r + 1) = \frac{1}{3} \cdot n(n + 1)(n + 2)$
    \begin{solution}
      % Enter your solution here.
    \end{solution}
    
  \part $1^5 + 2^5 + 3^5 + \ldots + n^5 = \frac{1}{12}  n^2 (n + 1)^2 (2n^2 + 2n - 1)$
    \begin{solution}
      % Enter your solution here.
    \end{solution}

  \part $5^{2n+1} + 2^{2n+1}$ is divisible by 7 for all $n \geq 0$.
    \begin{solution}
      % Enter your solution here.
    \end{solution}

    
  \part $a^{2^n} - 1$ is divisible by $8 \times 2^{n-1}$ for all odd integers $a$, and for all integers $n>0$.
    \begin{solution}
      % Enter your solution here.
    \end{solution}
    
  \part if $A_1, A_2,\ldots,A_n$ and $B$ are sets, then
    \[
      (A_1-B)\cup(A_2-B)\cup\ldots\cup(A_n-B)=(A_1\cup A_2\cup\ldots \cup A_n)-B
    \]
    \begin{solution}
      % Enter your solution here.
    \end{solution}
  \end{parts}

\question For which non-negative integers $n$ does the property $n^2 \le n!$ hold? Prove your answer using mathematical induction.
  \begin{solution}
    % Enter your solution here.
  \end{solution}

\question We are familiar with the Fibonacci numbers, defined as $f_{n+1} = f_n + f_{n-1}, f_0=1, f_1=1$ for all $n \geq 1$. The Fibonacci numbers turn out to have several interesting properties. Prove each of the following by mathematical induction.
  \begin{parts}
  \part $f_1 + f_2 + f_3 + \ldots + f_n = f_{n+2} - 1$.
    \begin{solution}
      % Enter your solution here.
    \end{solution}

  \part $f_2 + f_4 + f_6 + \ldots + f_{2n} = f_{2n+1} - 1$.
    \begin{solution}
      % Enter your solution here.
    \end{solution}
  \end{parts}

\question A \textit{convex polygon} is a polygon in which all the interior angles are smaller than $\pi$ radians. An $n$-gon is a polygon with $n$ sides. Give a proof by mathematical induction that the sum of the interior angles of a convex $n$-gon is $(n-2)\pi$ where $n\ge 3$. 

  \textit{Hint}: A convex $n$-gon can be divided into a convex $(n-1)$-gon and a triangle.
  \begin{solution}
    % Enter your solution here.
  \end{solution}
  

\end{questions}
\end{document}
%%% Local Variables:
%%% mode: latex
%%% TeX-master: t
%%% End: